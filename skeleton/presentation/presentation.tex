\documentclass{beamer}

\usepackage[polish]{babel}
\usepackage[utf8]{inputenc}
\usepackage[OT4]{fontenc}
\usepackage{beamerthemeshadow}
\usepackage{fancyvrb}
\usepackage{graphicx}
\usepackage{url}
\usepackage{shortcuts}

\beamertemplateballitem
\beamertemplatenumberedballsectiontoc

\hypersetup{%
  pdftitle={Tutul Prezentacji},%
  pdfauthor={Imie Nazwisko}%
}

\title[Tytuł]{Tytuł}

\author{Imie Nazwisko}

\institute{Zakład pracy}
\date{data}

\begin{document}

\AtBeginSection[]
{
  \begin{frame}
    \frametitle{Spis treści}
    \tableofcontents[currentsection]
  \end{frame}
}


\frame{\titlepage}

\section{Sekcja}

%%%%%%%%%%%%%%%%%%%%%%%%%%%%%%%%%%%%%%%%%%%%%%%%%%%%%%%%%%%%%%%%%%%%%%%
\begin{frame}[fragile]
	\frametitle{Tutuł ramki}
	\BB{Blok}
	\begin{itemize}
	\item Pierwszy Item
	\end{itemize}
	\BI
	\I item jakistam
	\EI
	\EB
\end{frame}

%%%%%%%%%%%%%%%%%%%%%%%%%%%%%%%%%%%%%%%%%%%%%%%%%%%%%%%%%%%%%%%%%%%%%%%
\begin{frame}[fragile]
	\frametitle{Przykłady funkcji}
	\begin{block}{W kilku językach programowania}
	\begin{verbatim}
double f(double x, double y) {
     return x + y;
}
	\end{verbatim}
	%\IMG{3cm}{obrazek.png}
	\end{block}
\end{frame}

\end{document}
